\section{限制与讨论}
\label{sec:limitation}尽管我们在提高开源 LLMs 数学推理能力方面取得了积极成果,但仍存在一些局限性。首先,由于计算资源的约束,未能对更大规模的模型进行全面测试,这可能影响了模型性能的最大化。其次,我们的研究主要集中在基础至中等难度的数学问题上,对于模型处理高级别复杂数学推理的能力仍有待深入探讨。此外,虽然我们提出的方法展示了良好的效果,但对于不同类型数学问题的有效性和泛化能力还需要更多的实证支持。

未来的研究可以从以下几个方向展开:一是继续扩大实验模型的规模,二是探索更多类型的数学问题以评估模型的广度和深度,三是开发更先进的数据增强和微调技术,四是考虑将外部工具如计算器或符号计算系统集成到模型中,以辅助解决更复杂的数学问题。